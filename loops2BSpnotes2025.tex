%%%%%% RVpreamble.tex
%%  version April 18, 2022

% table of contents for pre-amble
% prepreamble
% margins
% picture stuff  [ready to delete]
% point commands
% fonts
% symbols
% operatornames
% editing commands


%two arrows pointing to the right:
% $\xymatrix{ A \ar@< 2pt>[r]^\phi\ar@<-2pt>[r]_\psi & B}$
%bowed arrows
%$$
%\xymatrix{U \times \R^n \ar[d]_{\pi} \\
%U\ar@/_/[u]_{f_1, \dots, f_n} 
% }$$
%$$
%\xymatrix{
%Z \ar[dr] \ar[drr]^0 \ar@{.>}[d]_{\exists !} \\  A \ar[r]^i \ar@/_1pc/[rr]_0  &
% B \ar[r]^f & C \\ 
%}$$

%\begin{figure}[ht] \begin{center}
%\includegraphics[scale=0.3]{figures/fig82.jpeg}
%\end{center} \caption{} \label{} \end{figure}


%%%%%%%%%%%%%%%%%%%%%%%%%%%%%%%%%%%%%%%%%%%%%%%%%%%%%%%%%%%%%%%%%%%%%%%%%%%%%%%%%%%%%%%%%%%%%%%%%%%%%%%%%%%%%%%%%%%%%%%%%%
% prepreamble

\documentclass[12pt,letterpaper]{amsart} % LatexMasterTemplate May 2020 on overleaf
\usepackage{palatino, euler, epic,eepic, amssymb, xypic, floatflt, microtype}
\usepackage{verbatim,color}
\usepackage[linktocpage,hidelinks]{hyperref}



% June 9 09:  removed latexsym, amscd, and added palatino, microtype

\usepackage{tikz-cd}

\input xy
\xyoption{all}
% AK and TT say:  if your fonts are screwed up, remove the "renewcommand".
\renewcommand{\familydefault}{ppl}

%%%%%%%%%%%%%%%%%%%%%%%%%%%%%%%%%%%%%%%%%%%%%%%%%%%%%%%%%%%%%%%%%%%%%%%%%%%%%%%%%%%%%%%%%%%%%%%%%%%%%%%%%%%%%%%%%%%%%%%%%%
% margins

% from Allen K and Terry T's notices article
\setlength{\oddsidemargin}{0cm} \setlength{\evensidemargin}{0cm}
\setlength{\marginparwidth}{0in}
\setlength{\marginparsep}{0in}
\setlength{\marginparpush}{0in}
\setlength{\topmargin}{0in}
\setlength{\headheight}{0pt}
\setlength{\headsep}{0pt}
\setlength{\footskip}{.3in}
\setlength{\textheight}{9.2in}
\setlength{\textwidth}{6.5in}
\setlength{\parskip}{4pt}

\linespread{1.15} % following Ian Morrison

%%%%%%%%%%%%%%%%%%%%%%%%%%%%%%%%%%%%%%%%%%%%%%%%%%%%%%%%%%%%%%%%%%%%%%%%%%%%%%%%%%%%%%%%%%%%%%%%%%%%%%%%%%%%%%%%%%%%%%%%%%
% picture stuff
 
% \newlength{\baseunit}               % the basic unit length
% \newlength{\templengthhoriz}        % width of the picture
% \newlength{\templengthvert}         % depth of the picture
% \newlength{\temprule}               % with between left margin and picture
% \newcount{\numlines}                % depth of picture (in number of lines)
% \setlength{\baseunit}{0.05ex}
  
%  \newcommand{\getfig}[2] {
%           \setlength{\unitlength}{#2\baseunit}
%            \input #1.tex }
                         % A macro to input the picture.  Use
                         % \getfig{name}{scale}


%%%%%%%%%%%%%%%%%%%%%%%%%%%%%%%%%%%%%%%%%%%%%%%%%%%%%%%%%%%%%%%%%%%%%%%%%%%%%%%%%%%%%%%%%%%%%%%%%%%%%%%%%%%%%%%%%%%%%%%%%%
% point commands (and  theorem-related commands)


\newcommand{\bpf}{\noindent {\em Proof.  }}
\newcommand{\epf}{\qed \vspace{+10pt}}

\newcommand{\point}{\vspace{3mm}\par \noindent \refstepcounter{subsection}{\thesubsection.} }
\newcommand{\tpoint}[1]{\vspace{3mm}\par \noindent \refstepcounter{subsection}{\thesubsection.} 
  {\bf #1. ---} }
\newcommand{\epoint}[1]{\vspace{3mm}\par \noindent \refstepcounter{subsection}{\thesubsection.} 
  {\em #1.} }
\newcommand{\bpoint}[1]{\vspace{3mm}\par \noindent \refstepcounter{subsection}{\thesubsection.} 
  {\bf #1.} }

%Exercise
\newtheorem{eexercise}{Eexercise}[section]
%\newcounter{exercise}
\renewcommand{\theeexercise}{\thesection.\Alph{eexercise}}
\newcommand{\exercise}[1]{\vspace{3mm}\par\refstepcounter{eexercise}\noindent{\bf
    \theeexercise.}   {\sc #1.} }
\newcommand{\exercisedone}{ \vspace{2mm}}


% I haven't really used this.
\newcommand{\theorem}{\noindent {\bf \addtocounter{subsubsection}{1} \arabic{section}.\arabic{subsubsection}  Theorem.} }

% I tend not to use these, and instead use tpoint above.
\newtheorem{tm}{Theorem}[subsection]\newtheorem{pr}[tm]{Proposition}\newtheorem{lm}[tm]{Lemma}\newtheorem{co}[tm]{Corollary}\newtheorem{df}[tm]{Definition}


%%%%%%%%%%%%%%%%%%%%%%%%%%%%%%%%%%%%%%%%%%%%%%%%%%%%%%%%%%%%%%%%%%%%%%%%%%%%%%%%%%%%%%%%%%%%%%%%%%%%%%%%%%%%%%%%%%%%%%%%%%
%%% fonts

% miscellaneous

\newcommand{\uR}{\underline{\R}}
\newcommand{\M}{M}
\newcommand{\vv}{\vec{v}}

%\newcommand{\ut}[1]{\underset{\widetilde{ }}{#1}}

% tilde
\newcommand{\tK}{\tilde{K}}
\newcommand{\tA}{\tilde{A}}
\newcommand{\tX}{\tilde{X}}
\newcommand{\tY}{\tilde{Y}}
\newcommand{\tZ}{\tilde{Z}}
\newcommand{\tC}{\tilde{C}}
\newcommand{\tS}{\tilde{S}}

%  overline

\newcommand{\cmbar}{\overline{\cm}}
\newcommand{\Fpbar}{\overline{\F}_p}
\newcommand{\oz}{\overline{z}}
\newcommand{\ow}{\overline{w}}


\newcommand{\Qbar}{\overline{\mathbb{Q}}}
\newcommand{\Ebar}{\overline{E}}
\newcommand{\kbar}{\overline{k}}
\newcommand{\wbar}{\overline{w}}

% bb
\newcommand{\proj}{\mathbb P}

\newcommand{\C}{\mathbb{C}}
\newcommand{\F}{\mathbb{F}}
\newcommand{\G}{\mathbb{G}}
\renewcommand{\H}{\mathbb{H}}
\newcommand{\Z}{\mathbb{Z}}
\newcommand{\A}{\mathbb{A}}
\newcommand{\E}{\mathbb{E}}
\renewcommand{\L}{\mathbb{L}}
\newcommand{\Q}{\mathbb{Q}}
\newcommand{\W}{\mathbb{W}}
\newcommand{\bbS}{\mathbb{S}}
\newcommand{\R}{\mathbb{R}}
\def\CC{\mathbb{C}}
\def\DD{\mathbb{D}}
\def\EE{\mathbb{E}}
\def\FF{\mathbb{F}}
\def\GG{\mathbb{G}}
\def\HH{\mathbb{H}}
\def\II{\mathbb{I}}
\def\JJ{\mathbb{J}}
\def\KK{\mathbb{K}}
\def\LL{\mathbb{L}}
\def\MM{\mathbb{M}}
\def\NN{\mathbb{N}}
\def\OO{\mathbb{O}}
\def\PP{\mathbb{P}}
\def\QQ{\mathbb{Q}}
\def\RR{\mathbb{R}}
\def\SS{\mathbb{S}}
\def\TT{\mathbb{T}}
\def\UU{\mathbb{U}}
\def\VV{\mathbb{V}}
\def\WW{\mathbb{W}}
\def\XX{\mathbb{X}}
\def\YY{\mathbb{Y}}
\def\ZZ{\mathbb{Z}}

% scr and mathcal
\newcommand{\cA}{{\mathscr{A}}}
\newcommand{\cB}{{\mathscr{B}}}
\newcommand{\cC}{{\mathscr{C}}}
\newcommand{\cE}{{\mathscr{E}}}
\newcommand{\cF}{{\mathscr{F}}}
\newcommand{\cG}{{\mathscr{G}}}
\newcommand{\cI}{{\mathscr{I}}}
\newcommand{\cJ}{{\mathscr{J}}}
\newcommand{\cK}{{\mathscr{K}}}
\newcommand{\cLL}{{\mathscr{L}}}
\renewcommand{\cL}{{\mathscr{L}}}
\newcommand{\cm}{{\mathscr{M}}}
\newcommand{\cM}{{\mathscr{M}}}
\newcommand{\cP}{{\mathscr{P}}}
\newcommand{\cQ}{{\mathscr{Q}}}
\newcommand{\cS}{{\mathscr{S}}}
\newcommand{\cT}{{\mathscr{T}}}
\newcommand{\cu}{{\mathscr{U}}}
\newcommand{\cV}{{\mathscr{V}}}
\newcommand{\cW}{{\mathscr{W}}}
\newcommand{\cX}{{\mathscr{X}}}
\newcommand{\cY}{{\mathscr{Y}}}
\newcommand{\ct}{{\mathscr{T}}}
\newcommand{\oh}{{\mathscr{O}}}

\def\calF{\mathcal{F}}
\def\calG{\mathcal{G}}
\def\calH{\mathcal{H}}
\def\calI{\mathcal{I}}
\def\calJ{\mathcal{J}}
\def\calK{\mathcal{K}}
%\newcommand{\calL}{{\mathscr{L}}}
\def\calL{\mathcal{L}}
\def\calM{\mathcal{M}}
\def\calN{\mathcal{N}}
\def\calO{\mathcal{O}}
\def\calP{\mathcal{P}}
\def\calQ{\mathcal{Q}}
\def\calR{\mathcal{R}}
\def\calS{\mathcal{S}}
\def\calT{\mathcal{T}}
\def\calU{\mathcal{U}}
\def\calV{\mathcal{V}}
\def\calW{\mathcal{W}}
\def\calX{\mathcal{X}}
\def\calY{\mathcal{Y}}
\def\calZ{\mathcal{Z}}

% bf
\def\bU{\mathbf{U}}
\def\bV{\mathbf{V}}
\def\bW{\mathbf{W}}
\def\bX{\mathbf{X}}
\def\bY{\mathbf{Y}}
\def\bZ{\mathbf{Z}}
\newcommand{\D}{\mathbf{D}}

\def\bJ{\mathbf{J}}
\def\bK{\mathbf{K}}
\def\bL{\mathbf{L}}
\def\bM{\mathbf{M}}
\def\bN{\mathbf{N}}
\def\bO{\mathbf{O}}
\def\bP{\mathbf{P}}
\def\bQ{\mathbf{Q}}
\def\bR{\mathbf{R}}
\def\bS{\mathbf{S}}
\def\bT{\mathbf{T}}
\def\bU{\mathbf{U}}
\def\bV{\mathbf{V}}
\def\bW{\mathbf{W}}
\def\bX{\mathbf{X}}
\def\bY{\mathbf{Y}}
\def\bZ{\mathbf{Z}}


% frak

\newcommand{\fm}{\mathfrak{m}}
\newcommand{\fr}{\mathfrak{r}}
\newcommand{\fM}{{\mathfrak{M}}}
\newcommand{\fU}{{\mathfrak{U}}}

\newcommand\frA{\mathfrak{A}}
\newcommand\frB{\mathfrak{B}}
\newcommand\frg{\mathfrak{g}}
\newcommand\frt{\mathfrak{t}}
\newcommand\frb{\mathfrak{b}}
\newcommand\frh{\mathfrak{h}}
\newcommand\frn{\mathfrak{n}}
\newcommand\frN{\mathfrak{N}}
\newcommand\frl{\mathfrak{l}}
\newcommand\frp{\mathfrak{p}}
\newcommand\frq{\mathfrak{q}}
\newcommand\frr{\mathfrak{r}}
\def\frm{\mathfrak{m}}



%%%%%%%% tilde %%%%%%%%%


\newcommand\tilW{\widetilde{W}}
\newcommand\tilA{\widetilde{A}}
\newcommand\tilB{\widetilde{B}}
\newcommand\tilC{\widetilde{C}}
\newcommand\tilD{\widetilde{D}}
\newcommand\tilE{\widetilde{E}}
\newcommand\tilF{\widetilde{F}}
\newcommand\tilG{\widetilde{G}}

%%%%%%%% hat %%%%%%%%%

\def\hatG{\widehat{G}}
\newcommand\hA{\widehat{A}}
\newcommand\hLam{\widehat{\Lambda}}
\newcommand\hZ{\widehat{\ZZ}}



%%%%%%%%%%%%%%%%%%%%%%%%%%%%%%%%%%%%%%%%%%%%%%%%%%%%%%%%%%%%%%%%%%%%%%%%%%%%%%%%%%%%%%%%%%%%%%%%%%%%%%%%%%%%%%%%%%%%%%%%%%
% symbols


% greek 

\newcommand{\al}{\alpha}
\newcommand{\ka}{\kappa}

\newcommand{\om}{\omega}
\newcommand{\be}{\beta}
\newcommand{\ga}{\gamma}
\newcommand{\Ga}{\Gamma}
\newcommand{\de}{\delta}
\newcommand{\De}{\Delta}
\newcommand{\si}{\sigma}
\newcommand{\la}{\lambda}

% other

\newcommand{\dis}{\bullet}
\newcommand{\virt}{\rm{virt}}
\newcommand{\Kvar}{K_{(\rm{Var})}}
\newcommand{\fixed}{\rm{fixed}}
\newcommand{\Add}{\rm{Add}}
\renewcommand{\top}{\rm{top}}
\newcommand{\red}{\rm{red}}

\newcommand{\coloneq}{:=}



\newcommand{\propernormal}{%
  \mathrel{\ooalign{$\lneq$\cr\raise.22ex\hbox{$\lhd$}\cr}}}


%%%%%%%%%%%%%%%%%%%%%%%%%%%%%%%%%%%%%%%%%%%%%%%%%%%%%%%%%%%%%%%%%%%%%%%%%%%%%%%%%%%%%%%%%%%%%%%%%%%%%%%%%%%%%%%%%%%%%%%%%%
% operatornames

\newcommand{\opn}[1]{\operatorname{#1}}

%\newcommand{\spam}{\operatorname{span}}

% category theory
\newcommand{\cat}[1]{{\text{\underline{\bf #1}}}}
\newcommand{\colim}{\operatorname{colim}}
\newcommand{\Ext}{\operatorname{Ext}}
\newcommand{\Mor}{\operatorname{Mor}}
\newcommand{\Isom}{\operatorname{Isom}}
\newcommand{\same}{{\overset {\displaystyle \sim} \longleftrightarrow}}
\newcommand{\sametwo}{\xrightarrow{\sim}}

% group and ring theory
\newcommand\Stab{\operatorname{Stab}}
\newcommand{\Orb}{\operatorname{Orb}}
\newcommand{\Sym}{\operatorname{Sym}}
\newcommand{\Aut}{\operatorname{Aut}}
\newcommand{\rank}{\operatorname{rank}}
\renewcommand{\span}{\operatorname{span}}
\newcommand{\coker}{\operatorname{coker}}
\newcommand{\val}{\operatorname{val}}
\newcommand{\chr}{\operatorname{char}}
\newcommand{\Gal}{\operatorname{Gal}}
\newcommand{\Ad}{\operatorname{Ad}}
\newcommand{\uGal}{\underline{\Gal}}
\newcommand{\Hom}{\operatorname{Hom}}
\newcommand{\Fn}{\operatorname{Maps}}
\newcommand{\Maps}{\operatorname{Maps}}


% algebraic geometry
\newcommand{\codim}{\operatorname{codim}}
\newcommand{\Div}{\operatorname{Div}}
\newcommand{\Hilb}{\operatorname{Hilb}}
\newcommand{\Pic}{\operatorname{Pic}}
\newcommand{\Spec}{\operatorname{Spec}}
\newcommand{\Cl}{\operatorname{Cl}}
\newcommand{\res}{\operatorname{res}}
\renewcommand{\div}{\operatorname{div}}
\newcommand{\Trop}{\operatorname{Trop}}
\newcommand{\Jac}{\operatorname{Jac}}

\newcommand{\aff}{\rm{aff}}

\newcommand{\Gr}{\rm{Gr}}
\newcommand{\LGr}{\rm{LGr}}
\newcommand{\SAG}{\rm{SpAfGr}}
\newcommand{\LSG}{\rm{LgStGr}}

% define \jalpha so that it can be used in math mode
\newcommand{\jalpha}{\mathord{j\mkern-7mu\alpha}}
%\newcommand{\jalpha}{  \overset{\alpha}{j}}

%%%%%%%%%%%%%%%%%%%%%%%%%%%%%%%%%%%%%%%%%%%%%%%%%%%%%%%%%%%%%%%%%%%%%%%%%%%%%%%%%%%%%%%%%%%%%%%%%%%%%%%%%%%%%%%%%%%%%%%%%%
% editing commands


%\newcommand{\remind}[1]{{\bf[#1]}}
\newcommand{\remind}[1]{{\bf[{\large TODO:  } #1]}}
\newcommand{\lremind}[1]{{\bf[label:  #1]}}
%\newcommand{\notation}[1]{\footnote{\scriptsize #1}}
\newcommand{\notation}[1]{}
%\renewcommand{\remind}[1]{{}}
\renewcommand{\lremind}[1]{{}}

\newcommand{\ravi}[1]{{\bf [#1 --- Ravi]}}
\newcommand{\hidden}[1]{\footnote{Hidden:  #1}}
%\renewcommand{\hidden}[1]{}

\newcommand{\cut}[1]{}
%\newcommand{\cut}[1]{#1}

%%%%%%%%%%%%%%%%%%%%%%%%%%%%%%%%%%%%%%%%%%%%%%%%%%%%%%%%%%%%%%%%%%%%%%%%%%%%%%%%%%%%%%%%%%%%%%%%%%%%%%%%%%%%%%%%%%%%%%%%%%
% formatting commands

\newcommand{\leftbox}[1]{      \begin{flushleft}\fbox{ \parbox{25em}{
#1        }}\end{flushleft}
}\newcommand{\centerbox}[1]{      \begin{center}\fbox{ \parbox{25em}{
#1        }}\end{center}
}
\newcommand{\rightbox}[1]{      \begin{flushright}\fbox{ \parbox{25em}{
#1        }}\end{flushright}
}

%%%%%%%%%%%%%%%%%%%%%%%%%%%%%%%%%%%%%%%%%%%%%%%%%%%%%%%%%%%%%%%%%%%%%%%%%%%%%%%%%%%%%%%%%%%%%%%%%%%%%%%%%%%%%%%%%%%%%%%%%%
% end of preamble

\begin{document}
\pagestyle{plain}
\title{\Large{$\Omega^2(BSp)$ notes}}
\author{Ravi Vakil for Jim Bryan}
%\address{Dept. of Mathematics, Stanford University, Stanford CA~94305--2125}
%\email{rvakil@stanford.edu}
%\date{March 23 2021}
\date{\today .   {\em Last edited:}
  July 13, 2025. (keep current date here)  }
%\subjclass{Primary 14F30, Secondary 14F20. }
%\begin{abstract}
% \end{abstract}
\maketitle
\tableofcontents


{\parskip=12pt % closing bracket is just before the bibliography 


These are Ravi's notes, for Jim, but also to set some notation.

\epoint{The space $V$}

We start with a vector space $\boxed{V}$ with an alternating form. 
(I expect everything I type here to work fine for $O$ in place of $Sp$  with change of signs, but I stick to the symplectic case for concreteness and to avoid confusion.) 
Choosing a splitting $V = L \oplus L^*$ is called a polarization or Lagrangian splitting. Jim says it is a "Weinstein normal form".

More generally, we can work over an arbitrary base, and all statements will behave well with respect to base change.
So most generally, we work over $\Spec \Z$, and $V$ is a free sheaf on $\Spec \Z$ of rank $2n$, etc.  I will continue to use vector space language for simplicity.


\section{Background on the symplectic affine Grassmannian}


\epoint{The loop space of $V$}

The loop space of $V$ is
$$H = V((z)) \coloneq V[[z]] \oplus z^{-1} V[ z^{-1}].$$


We define
$H^+ = V[[z]]$ and $H^- = z^{-1} V[ z^{-1}]$.
$H$ has a residue pairing:  
$$\langle f(z), g(z) \rangle \coloneq \operatorname{Res}_{z=0}  \omega(f(-z), g(z)) \; dz.$$
This residue pairing is skew-symmetric and nondegenerate.
$H$ is thought of as some sort of infinite-dimensional symplectic Hilbert space.


{\em Question:} what's the reason for the sign in pairing?  
I think in our pairing (see later on), there is no sign.


\epoint{The Lagrangian Sato Grassmannian} 

The Lagrangian Sato Grassmannian parametrizes "Lagrangians" of the loop space that are close to $H^+$.
It is contained in the index zero component of the (usual --- not symplectic) Sato Grassmannian (not defined here).
The Lagrangian Sato Grassmannian  is often denoted $\LGr_\infty$ (or $LGr(H)$).  I will use temporary notation $\LSG$ to remind us of its meaning. (This is a macro which can be easily changed.)


The $N$th truncation is denoted $\LSG^{(N)} = \LSG^{(N)}(H)$, and is isomorphic to $$LG(N \dim V, 2N \dim V).$$
We can show that $\LSG_\infty$ is the limit of $LGr(nN, 2nN)$ in our naive homotopy category.   Thus its homotopy type is $LGr(\infty, 2 \infty) = U/O = \Omega(Sp)$.

Remark:
This will likely be relevant for us, because we will be "increasing $V$" which will be changing the symplectic affine Grassmannian, but the ambient Lagrangian Sato Grassmannian (even for finite $V$) will be what we want.  

\epoint{The symplectic affine Grassmannian}

The symplectic affine Grassmannian is the $z$-stable locus inside the Lagrangian Sato Grassmannian.
The symplectic affine Grassmannian is  unfortunately 
often denoted  $Gr$ or $Gr_{Sp_{2n}}$ or $Gr_{Sp}$. I will use temporary notation $\SAG$ to remind us.  This is a temporary macro which can be easily changed.

Fact:  $\SAG_{Sp(V)}$ is homotopic to $\Omega(Sp(V))$.

Vague argument (that I've not thought through):    $L Sp(V)$ parametrizes maps from loops to $Sp(V)$ --- smooth maps from $S^1$.  $L^+(V)$ is the subset where the map extends holomorphically to the disk.  Then show that $L/L^+ \cong LGr$, by describing a transitive action of $L$, and identifying the stabilizer as $L^+$.  But $L^+$ is contractible.  (Presumably I should look at \cite{ps} for more. It might be in a later paper of  Nadler, perhaps \cite{nadler}, see one of the references of \cite{zhu}.)   (There is an argument of this sort in the algebraic category, involving algebraic loops; and also in the continuous category.)

\epoint{Truncations of the symplectic affine Grassmannian}
The $N$th truncation of the symplectic affine Grassmannian is 
called $\SAG_{Sp}^{(N)} = \Gr^{(N)}$.  (Caution:  I am worried that the terminology  $\Gr$  gets used both for the affine and the Sato case.)

\epoint{Singularities of the truncation}
Apparently $\SAG^{(N)}$  is singular in codimension $\dim V$, with possible reference  \cite{mov}. 


\section{Moduli of vector bundles}

Now let $\Omega^2_{alg}(BSp(2n))$  be the moduli space of vector bundles on $\proj^1$, framed at $\infty$ by $V$, with its form.

Let $\Omega^2_{alg}(BSp(2n))^{[N]}$ parametrize those bundles $\cF$ such that $\cF(N)$ is globally generated. For example, 
$\Omega^2_{alg}(BSp(2n))^{[N]}$ is empty if $N<0$.

\epoint{Basic facts about this space}
The space 
$\Omega^2_{alg}(BSp(2n))$ 
is an Artin stack.  

 The space 
$\Omega^2_{alg}(BSp(2n))$ is an $Sp(2n)$ bundle over the "unframed" moduli space, which is also thus an Artin stack.  This latter space is smooth, because the automorphisms/deformation/obstructions of a principle bundle $E$ is given by the cohomology of the adjoint bundle $ad(E)$, so automorphisms are $H^0(ad(E))$, deformations are $H^1(ad(E))$, and obstructions are $H^2(ad(E))$, which in this case are zero.  (The adjoint bundle is the "twisted Lie algebra bundle", as I think Jim was telling me.)



The space 
$\Omega^2_{alg}(BSp(2n))$  is the union of an increasing sequence of open substacks
$\Omega^2_{alg}(BSp(2n))^{[N]}$.

Each $\Omega^2_{alg}(BSp(2n))^{[N]}$ is quasicompact and finite type.  

$\Omega^2_{alg}(BSp(2n))^{[0]}$ is a reduced point.

The space of  bundles trivialized in a formalized neighborhood of $p_\infty$ is apparently, as an ind-scheme, the affine symplectic Grassmannian $\SAG{Sp(2n)}$.

\epoint{Sketch of why these things are true}
These are all well-known facts, but we will also end reproving them, so we can have confidence in these statements.  Here is the architecture of our argument.

First:
$\Omega^2_{alg}(BSp(2n))$ and $\Omega^2_{alg}(BSp(2n))^{[N]}$
are stacks in the usual (smooth etc.) topology (not yet obviously algebraic stacks).  

(ii) 
We construct a finite type affine scheme (explicitly, with generators and relations), that will parametrize the same things as  $\Omega^2_{alg}(BSp(2n))^{[N]}$, with in addition a Zariski-splitting.  This is an affine bundle  over $\Omega^2_{alg}(BSp(2n))^{[N]}$.  Thus $\Omega^2_{alg}(BSp(2n))^{[N]}$ is an algebraic stack (i.e., an Artin stack).

We  have open embeddings of these truncations, and their union is  the entire space, so $\Omega^2_{alg}(BSp(2n))$ is an algebraic stack.




$\Omega^2_{alg}(B Sp(V))$ parametrizes the following.

\begin{itemize}
\item $\boxed{\cF}$ is a rank $2n$ vector bundle on $\proj^1$.  (Caution:  earlier the bundle on $\proj^1$ was considered to be rank $n$.)
\item  We have an identification $\cF|_{p_{\infty}} \overset \sim \longrightarrow V$
\item $\boxed{\phi_\cF}: \cF \overset \sim \longrightarrow \cF^\vee$ satisfying 
$\phi^\vee_\cF = -\phi_\cF$, and $\phi_{\cF}|_{p_\infty} = \phi_V$ (where $\phi_V$ comes
from the alternating form).    Or equivalently:  $\boxed{\psi_{\cF}}: \cF\otimes \cF \rightarrow \oh$,
with appropriate hypotheses.
\end{itemize}


\epoint{Comparison to topology}
I think we quote Cohen-Lupercio-Segal or someone else to show that smooth maps to $BSp(V)$ are homotopic to holomorphic maps to $BSp(V)$.
Then by GAGA (some justification needed) this is the same as algebraic maps to $BSp(V)$.

\point $\boxed{\Omega^2_{d,alg}(B Sp(V))}$ parametrizes the same, with the additional requirement that $\cF(d p_\infty)$ is
generated by global sections.  (Equivalently, when you write $\cF$ as a direct sum of line bundles, the summands
are all of degree between $-d$ and $d$ inclusive.  This interpretation is {\em not helpful}.)

We define $\boxed{\cE} = \cF (d p_{\infty})$ for convenience.  This bundle is rank $2n$ and 

We then have $\boxed{\phi_{\cE}} : \cE \rightarrow \cE^\vee(2d)$, and $\boxed{\psi_{\cE}} = \psi_{\cF}(2d) : \cE \otimes \cE \rightarrow \oh(2d p_{\infty})$.

\tpoint{Claim}
{\em We have an induced isomorphism $\cE|_{p_{\infty}} \overset \sim \longrightarrow V$.}

(Proof omitted.)

\epoint{Zariski-framing}

Define $\boxed{A}  \coloneq H^0(\cE(-p_\infty))$, which has dimension $2dn$.  

We now consider the Zariski-framed moduli space, which doesn't yet have a name.

The data of the Zariski-framed bundle $\cE$ is the data of $A$, plus $\al: A \rightarrow A$, and $j:A \rightarrow U$,
along with an {\em open condition} on $j$ and $\al$.  I've lost track, but I think the condition is codimension $2n$.


For future use, define $\jalpha$ (need better name) by...


}  %end parskip at the start of the file


\begin{thebibliography}{[HPLqjm]}


\bibitem[MOV]{mov} A. Malkin, V. Ostrik, and M. Vybornov, {\em  The minimal degeneration singularities in the affine Grassmannians}, Duke Math. J., 126(2):233–249,
2005.

\bibitem[Na]{nadler} David Nadler, {\em Matsuki correspondence for the affine Grassmannian}, Duke Math. J. 124 (2004), no.\ 3,
421–457.

\bibitem[PS]{ps} 
Andrew Pressley and Graeme Segal, Loop groups, Oxford Mathematical Monographs, The Clarendon Press, Oxford University Press, New York, 1986. 

\bibitem[Z]{zhu} Xinwen Zhu, {\em An introduction to affine Grassmannians and the geometric Satake equivalence}, IAS/Park City Mathematics Series (expanded lecture notes from the 2015 PCMI Summer School), April 4, 2016.  \verb+https://arxiv.org/abs/1603.05593+
\end{thebibliography}

\end{document}
