\documentclass{amsart}

\title{Based maps to Lagrangian Grassmanians and Quivers}
\author{Jim Bryan and Ravi Vakil}
\date{Last compiled:  \today.  Last edited:  December 12, 2024.}
\address{
Department of Mathematics\\
University of British Columbia \\
Room 121, 1984 Mathematics Road  \\
Vancouver, B.C., Canada V6T 1Z2  
}



\usepackage{diagbox}
\usepackage{float}
\usepackage{graphicx}
\usepackage{booktabs}
\usepackage{amsmath}
\usepackage{bm}
%\usepackage{pict2e,keyval,calc,fp}
%\usepackage{etex}
\usepackage{verbatim}
\usepackage{amsmath,amsthm,amsfonts}
\usepackage{amssymb}
\usepackage{times}
\usepackage{longtable}
\usepackage{caption}
\usepackage{array}
\usepackage{tikz-cd}
\usepackage{mathtools}
%\usepackage{amstex}
%\linespread{1.1}



\newtheorem{theorem}{Theorem}[section]
\newtheorem{proposition}[theorem]{Proposition}
\newtheorem{conjecture}[theorem]{Conjecture}
\newtheorem{lemma}[theorem]{Lemma}
\newtheorem{corollary}[theorem]{Corollary}
\newtheorem{convention}{Convention}[theorem]
\newtheorem{case}{Case}[theorem]
%\numberwithin{case}{theorem}


\theoremstyle{definition}

\newtheorem{remark}[theorem]{Remark}
\newtheorem{def-theorem}[theorem]{Definition-Theorem}
\newtheorem{definition}[theorem]{Definition}
\newtheorem{example}[theorem]{Example}
\newtheorem{assumption}[theorem]{Assumption}





\usepackage{amsmath}
\usepackage{amsmath,amsthm,amsfonts}
\usepackage{times}
%\usepackage{amstex}

\newcommand{\smargin}[1]{\marginpar{\tiny{#1}}}

\newcommand{\CC} {{\mathbb C}}          % complex numbers
\newcommand{\NN} {{\mathbb N}}		% natural numbers
\newcommand{\RR} {{\mathbb R}}		% real numbers
\newcommand{\ZZ} {{\mathbb Z}}		% integers
\newcommand{\QQ} {{\mathbb Q}}		% rationals
\newcommand{\FF} {{\mathbb F}}
\newcommand{\evir}{e_{\mathsf{vir}}}
\newcommand{\reg}{\mathsf{reg}}
\newcommand{\calE}{\mathcal{E}}


\newcommand{\PP}{\mathbb{P}}
\newcommand{\OO}{\mathcal{O}}
\newcommand{\M}{{\mathsf{M}}}


\newcommand{\tinyhalf}{\tfrac{1}{2}}

\newcommand{\Hom}{\operatorname{Hom}}
\newcommand{\Ker}{\operatorname{Ker}}
\newcommand{\End}{\operatorname{End}}
\newcommand{\Ext}{\operatorname{Ext}}
\newcommand{\Tr}{\operatorname{tr}}
\newcommand{\tr}{\operatorname{tr}}
\newcommand{\Coker}{\operatorname{Coker}}
\newcommand{\coker}{\operatorname{coker}}
\newcommand{\im}{\operatorname{Im}}
\newcommand{\Km}{\operatorname{Km}}
\newcommand{\Fun}{\operatorname{Fun}}
\newcommand{\Sym}{\operatorname{Sym}}
\newcommand{\NS}{\operatorname{NS}}
\newcommand{\Pic}{\operatorname{Pic}}
\newcommand{\Coef}{\operatorname{Coef}}
\newcommand{\Isom}{\operatorname{Isom}}
\newcommand{\Hilb}{\operatorname{Hilb}}
\newcommand{\Tot}{\operatorname{Tot}}
\newcommand{\UL}[1]{\underline{#1}}
\newcommand{\alg}{\mathsf{alg}}
\renewcommand{\top}{\mathsf{top}}
\newcommand{\Gr}{\operatorname{Gr}}
\newcommand{\LoopTwo}{\Omega^{2}_{d,\alg}}
\newcommand{\LoopTwoTop}{\Omega^{2}_{d,\top}}
\newcommand{\Id}{\operatorname{Id}}
\newcommand{\codim}{\operatorname{codim}}

\begin{document}

\begin{abstract}
  We study based algebraic maps from $\PP^{1}$ to the Lagrangian Grassmianian.
  And one more change.
\end{abstract}

\maketitle 

\markboth{Short title 1}  {Short title 2}
%\renewcommand{\sectionmark}[1]{}


%\tableofcontents
%\pagebreak


\section{A Retelling of Larson-Vakil}

This section probably won't end up in the paper. The purpose here is
to use it as the prototype for the way we treat the 8-fold case and to
fix some ideas and notation.

\subsection{Notation and Definitions}

We fix homogeneous coordinates $(x_{0}:x_{1})$ on $\PP^{1}$ and we let
$p_{\infty }=(1:0)$. Let $V$ be a vector space. We use the notation
\[
\UL{V}(k) = \OO_{\PP^{1}}(k)\otimes V.
\]

\begin{definition}\label{defn: Omega2alg}
Let $(X,\bullet )$ be a (quasi-projective) variety with a base point
$\bullet \in X$. We define the algebraic loop-2 space of $X$ by
\[
\LoopTwo (X) = \left\{f:\PP^{1}\to X \quad :\quad
f(p_{\infty})=\bullet , f_{*}([\PP^{1}])=d\in A_{1}(X) \right\}.
\]
\end{definition}


Let $U$ and $W$ be vector spaces of dimension $n$ and $N$ respectively
and let
\[
\Gr (n,n+N) = \Gr (n,U\oplus W)
\]
be the Grassmannian of $n$ dimensional quotients of $U\oplus W$ with
basepoint given by the projection $U\oplus W\to U$. By pulling back
the universal quotient, we see that $\LoopTwo (\Gr (n,n+N))$
parameterizes exact sequences:

\[
\begin{tikzcd}
0\arrow[r]& F \arrow[r]& \UL{U}\oplus \UL{W} \arrow{r}{\theta } &E
\arrow[r]& 0
\end{tikzcd}
\]
where $E$ is a rank $n$, degree $d$ bundle on $\PP^{1}$ and
$E|_{p_{\infty}}\cong U$ via the first component of $\theta
=({\UL{\theta}_U},\UL{\theta}_{W})$ and where the map
\[
\UL{\theta }_{{W}}:\UL{W}\to E
\]
is zero at $p_{\infty}$ and hence
factors through a map
\[
\UL{\gamma} :\UL{W} \to E(-p_{\infty}).
\]

Since $E$ is a quotient of a trivial bundle, it is non-negative in the
following sense:
\begin{definition}
A vector bundle $E$ on $\PP^{1}$ is non-negative (respectively
positive, less than $k$, etc.) if its splitting type $E\cong
\oplus_{i}\OO (a_{i})$ satisfies $a_{i} \geq 0$ (respectively
$a_{i}>0$, $a_{i}<k$, etc.) for all $i$. 
\end{definition}


Because $E$ is non-negative, the sheaf map $\UL{\theta}_{U}$ is
determined by its map on global sections which we denote without the
underline:
\[
\theta_{U}:U \longrightarrow H^{0}(E).
\]
Then the exact sequence
\[
\begin{tikzcd}
0\arrow[r]&E(-p_{\infty})\arrow{r}{\cdot \UL{x}_{1}}
&E\arrow[r]&E|_{p_{\infty}} \arrow[r] &0
\end{tikzcd}
\]
induces a sequence
\[
\begin{tikzcd}
0\arrow[r]&H^{0}(E(-p_{\infty}))\arrow{r}{\cdot {x}_{1}}
&H^{0}(E)\arrow[r]&U \arrow[r] &0
\end{tikzcd}
\]
which is split by $\theta_{U}:U\to H^{0}(E)$ inducing the isomorphism
\[
H^{0}(E) = A\oplus U
\]
where 
\[
A=H^{0}(E(-p_{\infty}))
\]
is a vector space of dimension $d$.

Then the sheaf map
\[
\begin{tikzcd}
 E(-p_{\infty})\arrow{r}{\cdot \UL{x}_{0}} &E
\end{tikzcd}
\]
induces a map in cohomology
\[
\begin{tikzcd}
A\arrow{r}{\cdot x_{0}} &A\oplus U
\end{tikzcd}
\]
whose components we denote by $\alpha$ and $j$:
\[
\alpha :A\to A, \quad j: A\to U.
\]
Finally, the sheaf map
\[
\begin{tikzcd}
\UL{W}\arrow{r}{\UL{\gamma}} & E(-p_{\infty})
\end{tikzcd}
\]
is determined by its map in cohomology
\[
\gamma :W\to A.
\]


\subsection{The main result}
With the notation of the previous subsection we can now state the main
geometric result of this section.

\begin{theorem}\label{thm: quiver description of loop2 of Gr}
Let $U$, $W$, and $A$ be vector spaces of dimension $n$, $N$, and
$d$. Let $V = \Hom (A,A)\oplus \Hom (A,U)\oplus \Hom (W,A)$ and let
$V^{0}\subset V$ be the open set of $(\alpha ,j,\gamma )\in V$
satisfying
\begin{enumerate}
\item $\Ker (\alpha -\lambda \cdot \Id_{A})\cap \Ker (j) = 0$ for all
$\lambda$.
\item  $\im (\alpha -\lambda \cdot \Id_{A})+ \im (\gamma )=A$ for
all $\lambda$.
\end{enumerate}
Then the constructions of the previous subsection induce an
isomorphism of affine varieties:
\[
\LoopTwo (\Gr (n,n+N))\cong V^{o}/GL(A) .
\]
\end{theorem}
\begin{remark}\label{rem: Gr(n,n+N) is a model for BGL(n) and Vo/GL(A) is a model for BGL(d)}
The Grassmannian on the left is an approximate model for $BGL(n)$ and
the quotient on the right is an approximate model for $BGL(d)$. We can
thus regard our theorem as saying, in a sense that we will make
precise in the next subsection, that there is an approximate
equivalence $\Omega^{2}(BGL(n))\sim BGL(d)$.
\end{remark}

\begin{remark}\label{rem: duality induced by Gr(n,n+N)=Gr(N,n+N)}
The isomorphism $\Gr (n,U\oplus W)\cong \Gr (N,U^{\vee}\oplus
W^{\vee})$ given by dualizing exact sequences induces an isomorphism
on the quiver side given by $(A,U,W)\mapsto
(A^{\vee},W^{\vee},U^{\vee})$ and $(\alpha ,j,\gamma )\mapsto
(\alpha^{\vee},\gamma^{\vee},j^{\vee})$. Note that the two open
conditions (1) and (2) are dual to each other under this equivalence. 
\end{remark}



To prove the theorem we begin with the following
\begin{lemma}\label{lem: Stromme}
Let $E$ be a non-negative bundle on $\PP^{1}$. Then the following
sequence is exact:
\[
\begin{tikzcd}
0\arrow{r} & \UL{H^{0}(E(-1))}(-1)
\arrow{rr}{\UL{x}_{1}x_{0}-\UL{x}_{0}x_{1}}& &
\UL{H^{0}(E)} \arrow{r} & E\arrow[r] & 0
\end{tikzcd}
\]
where as in the previous section, sheaf maps are denoted with
underlines and their induced maps on cohomology are denoted without,
so in particular $\UL{x}_{i}:\OO (-1)\to \OO$ and
$x_{i}:H^{0}(E(-1))\to H^{0}(E)$ in the above.
\end{lemma}
\begin{proof}
Let $\PP^{1}\times \PP^{1}$ have coordinates
$((y_{0}:y_{1}),(x_{0}:x_{1}))$ with $p,q:\PP^{1}\times \PP^{1}\to
\PP^{1}$ projection onto the first and second factor respectively. Let 
\[
\Delta =\{y_{1}x_{0}-y_{0}x_{1} =0 \}\subset \PP^{1}\times \PP^{1}
\]
be the diagonal. Then we get an isomorphism
\[
E\cong q_{*}\left(p^{*}E\otimes \OO_{\Delta} \right)
\]
where we have implicitly identified the two factors of $\PP^{1}$ by
$x_{i}=y_{i}$.  We then tensor the exact sequence
\[
\begin{tikzcd}
0\arrow[r] & \OO_{\PP^{1}\times \PP^{1}}(-1,-1)
\arrow{rr}{y_{1}x_{0}-y_{0}x_{1}}& & \OO_{\PP^{1}\times \PP^{1}}
\arrow[r]& \OO_{\Delta} \arrow[r]& 0
\end{tikzcd}
\]
by $p^{*}E$ and apply the functor $q_{*}(-)$. Since the non-negativity
of $E$ implies that $R^{1}q_{*}(p^{*}E(-1,-1))=0$, we get
\[
\begin{tikzcd}
0\arrow[r] &q_{*}(p^{*}E(-1,-1)) \arrow[r] & q_{*}(p^{*}E) \arrow[r] &
q_{*}(p^{*}(E))\otimes \OO_{\Delta} \arrow[r] & 0 
\end{tikzcd}
\]
which can be written as 
\[
\begin{tikzcd}
0\arrow[r] &\UL{H^{0}(E(-1))}(-1)
\arrow{rr}{\UL{y_{1}}x_{0}-\UL{y_{0}}x_{1}} && \UL{H^{0}(E)} \arrow[r]
&E \arrow[r] &0
\end{tikzcd}
\]
which proves the lemma.
\end{proof}

Now let 
\[
\UL{U} \oplus \UL{W} \to E \to 0
\]
be a point in $\LoopTwo (\Gr (n,n+N))$ and recall that 
\[
x_{0},x_{1}: H^{0}(E(-p_{\infty})) \to H^{0}(E)
\]
are given by
\[
\begin{tikzcd}
A \arrow{r}{\begin{pmatrix} \alpha \\ j  \end{pmatrix}} &A\oplus U,
\quad & A \arrow{r}{\begin{pmatrix} \Id_{A} \\ 0  \end{pmatrix}} &A\oplus U
\end{tikzcd}
\]
respectively. Then the exact sequence from Lemma~\ref{lem: Stromme}
reads
\[
\begin{tikzcd}
0 \arrow[r] &\UL{A}(-1) \arrow{rr}{\begin{pmatrix} \UL{x}_{1}\alpha -\UL{x}_{0}\\
\UL{x}_{1}j  \end{pmatrix}}&&\UL{A}\oplus \UL{U}\arrow[r]&E
\arrow[r]& 0.
\end{tikzcd}
\]

The condition that $\UL{A}(-1)$ is a subbundle of $\UL{A}\oplus
\UL{U}$ requires that for any point $\lambda =\frac{x_{0}}{x_{1}}\in
\PP^{1}-p_{\infty}$, the vector space map ${\tiny \begin{pmatrix} \alpha -\lambda \Id_{A}\\
j \end{pmatrix}} : A\to A\oplus U $ is injective. This is equivalent
to condition (1) of the theorem.

The condition that $\theta :\UL{U}\oplus \UL{W} \to E$ is surjective
means that the map 
\[
\UL{W} \to Q
\]
is surjective where $Q$ is the rank zero sheaf given by the cokernal
of $\UL{\theta}_{U}$:
\[
\begin{tikzcd}
0\arrow[r]& \UL{U} \arrow{r}{\UL{\theta}_{U}} & E\arrow[r]&Q\arrow[r]&0.
\end{tikzcd}
\]
The cohomology exact sequence of the above canonically identifies
$A\cong H^{0}(Q)$ and under this identification, the map in cohomology
induced by $\UL{W}\to Q$ is just $\gamma :W\to A$.

\emph{Sketchy completion of this argument:}  The support of $Q$ is
given by the eigenvalues of $\alpha$ so $\UL{W}\to Q$ being surjective
is equivalent to $\gamma$ surjecting onto the eigenspaces of $\alpha$
which is exactly condition (2) of the theorem. This more or less finishes up the
proof of the theorem. I guess I haven't said very carefully that the
point in $\LoopTwo (\Gr (n,n+N))$ is uniquely determined by $(\alpha
,j,\gamma )$ up to the action of $GL(A)$ although it is implicit in
the above argument. Also needed: prove that the quotient $V^{o}/GL(A)$
is an affine variety (it is an affine GIT  quotient). \emph{Is this
even true? I think it is.}


\subsection{Topological consequences of the theorem}\label{subsec:
topological consequences}

In this subsection, we make precise the assertion in Remark~\ref{rem:
Gr(n,n+N) is a model for BGL(n) and Vo/GL(A) is a model for
BGL(d)}. We work over $\CC$ and we give our varieties the analytic
topology. Then since $\PP^{1}$ is homeomorphic to the 2-sphere
$S^{2}$, there is a natural inclusion
\[
\LoopTwo (\Gr (n,n+N))\hookrightarrow \LoopTwoTop  (\Gr (n,n+N))
\]
where the space on the right is the usual loop-2 space of continuous
based maps $f:S^{2}\to \Gr (n,n+N)$.

In \cite{Mann-Milgram-93}, Mann and Milgram study the above inclusion
and prove that it induces a homotopy equivalence through dimension
$2d+1$ (and that moreover this bound is sharp). In other words we have
an isomorphism of homotopy groups for all $k<2d+1$:
\begin{align*}
\pi_{k}(\LoopTwo (\Gr (n,n+N))) &\cong \pi_{k}(\LoopTwoTop (\Gr (n,n+N)))\\
&\cong \pi_{k+2}(\Gr (n,n+N)).
\end{align*}

Without loss of generality, we now assume that $N\geq n$. We then can
easily prove that 
\[
\pi_{k+2}(\Gr (n,n+N))\cong \pi_{k+1}(U(n)) \text{ for all $k<2n-1$}
\]
using the long exact homotopy sequences associated to the standard
fibrations 
\begin{align*}
U(n-1)\hookrightarrow &U(n) \to S^{2n-1}\\
U(n)\times U(N) \hookrightarrow & U(n+N)  \to \Gr (n,n+N).
\end{align*}

Combining the above we have
\begin{equation}\label{eqn: pik(loop2alg)=pik+1(U(n))}
\pi_{k}(\LoopTwo (\Gr (n,n+N))) \cong \pi_{k+1}(U(n)) \text{  for all
} k<\min (2n-1,2d+1).
\end{equation}


On the other hand, we may analyze the homotopy groups of the quiver
variety $V^{o}/GL(A)$ using the long exact homotopy sequence of the
fibration
\[
GL(A)\hookrightarrow V^{o} \to V^{o}/GL(A). 
\]
Now $V^{o} = V-Z$ and the inclusion $V^{o} \hookrightarrow V$ induces
isomorphisms on $\pi_{k}$ for all $k< 2\codim(Z)-1$
\cite[Prop~7.1]{Anderson-Fulton} and so the long exact sequence
\[
\pi_{k}(V^{o})\to \pi_{k}(V^{o}/GL(A)) \to \pi_{k-1}(GL(A)) \to \pi_{k-1}(V^{o})
\]
gives isomorphisms
\begin{equation}\label{eqn: pik(Vo/GL(A)) = pik-1(GL(A))}
\pi_{k}(V^{o}/GL(A))\cong \pi_{k-1}(GL(A)) \text{  for all $k<2\codim Z
-1$}.
\end{equation}
Then since $GL(A)$ is homotopy equivalent to $U(d)$,
equations~\eqref{eqn: pik(Vo/GL(A)) = pik-1(GL(A))} and \eqref{eqn:
pik(loop2alg)=pik+1(U(n))}, along with Theorem~\ref{thm: quiver
description of loop2 of Gr} yields
\begin{equation}\label{eqn: main homotopy equivalence with one
superfluous condition}
\pi_{k+1}(U(n))\cong \pi_{k-1}(U(d))\text{   for all   } k\leq \min
(2n-1,2d+1, 2\codim Z - 1).
\end{equation}

\begin{lemma}
$\codim Z \geq N$. 
\end{lemma}
\begin{proof}
\emph{(Sketch)} By definition $Z=Z_{1}\cap Z_{2} $ where
\begin{align*}
Z_{1}&= \left\{(\alpha ,j,\gamma ): \text{$\Ker j$ contains an
eigenvector of $\alpha$}.  \right\}\\
Z_{2}&= \left\{(\alpha ,j,\gamma ): \text{$\Coker \gamma  $ contains an
eigenvector of $\alpha$}.  \right\}.
\end{align*}
The codimension of $Z_{1}\subset V$ is the same as the codimension of
$\Ker j \subset A$ which is $n$. By duality, the codimension of
$Z_{2}$ is $N$. The expected codimension of $Z$ is then $n+N$, but we
can certainly conclude that $\codim Z\geq \max (n,N)=N$.
\end{proof}
The Lemma means that $2n-1\leq 2\codim Z-1$ and so \eqref{eqn: main
homotopy equivalence with one superfluous condition} simplifies to
give us the main topological consequence of Theorem~\ref{thm: quiver
description of loop2 of Gr}: 
\begin{proposition}\label{prop: main homotopy equivalence}
The isomorphism of varieties in Theorem~\ref{thm: quiver description
of loop2 of Gr} induces homotopy equivalences:
\[
\pi_{k+1}(U(n))\cong \pi_{k-1}((U(d))
\]
for all $k\leq \min (2n-1,2d+1)$.
\end{proposition}
The above then immediately implies the classical Bott periodicity for
$U=\lim_{n\to \infty}U(n)$:
\begin{corollary}
$\pi_{k+1}(U)\cong \pi_{k-1}(U)$ for all $k$ and moreover,
$\pi_{k}(U(n))\cong \pi_{k}(U) $ for all $k\leq 2n$.
\end{corollary}




\bibliography{BV-biblio}
\bibliographystyle{plain}

\end{document}

